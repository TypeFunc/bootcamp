% arara: pdflatex
% arara: bibtex
% arara: pdflatex
% arara: pdflatex
%\documentclass[11pt]{article}
\documentclass{article}

\usepackage{lmodern}
\usepackage{microtype}

% Package for customizing page layout
%\usepackage[letterpaper]{geometry}
\usepackage[
top    = 3cm,
bottom = 3cm,
left   = 4cm,
right  = 4cm]{geometry}
%\usepackage[letterpaper]{geometry}
% \usepackage{fullpage}

%%% This file can never be completed.
%%% If you need something but cannot find it,
%%% contact the TA Favonia!

%%%%%%%%%%%%%%%%%%%%%%%%%%%%%%%%%%%%%%%%
% Basic packages
%%%%%%%%%%%%%%%%%%%%%%%%%%%%%%%%%%%%%%%%
\usepackage{amsmath,amsthm,amssymb}
\usepackage{mathtools}
\usepackage{etoolbox}
\usepackage{fancyhdr}
\usepackage{mathpartir}
\usepackage{xcolor}
\usepackage{hyperref}
\usepackage{xspace}
\usepackage{comment}
\usepackage{url} % for url in bib entries

%%%%%%%%%%%%%%%%%%%%%%%%%%%%%%%%%%%%%%%%
% Acronyms
%%%%%%%%%%%%%%%%%%%%%%%%%%%%%%%%%%%%%%%%
\usepackage[acronym, shortcuts]{glossaries}

\newacronym{HoTT}{HoTT}{homotopy type theory}
\newacronym{IPL}{IPL}{intuitionistic propositional logic}
\newacronym{TT}{TT}{intuitionistic type theory}
\newacronym{LEM}{LEM}{law of the excluded middle}
\newacronym{ITT}{ITT}{intensional type theory}
\newacronym{ETT}{ETT}{extensional type theory}
\newacronym{NNO}{NNO}{natural numbers object}

% Make \ac robust.
\robustify{\ac}

%%%%%%%%%%%%%%%%%%%%%%%%%%%%%%%%%%%%%%%%
% Fancy page style
%%%%%%%%%%%%%%%%%%%%%%%%%%%%%%%%%%%%%%%%
\pagestyle{fancy}
\newcommand{\metadata}[2]{
  \lhead{Algebra and Logic Seminar}
  \chead{IPL Notes}
  \rhead{February 28, 2014}
  %% \lfoot{#1}
  %% \cfoot{#2}
  %% \rfoot{\thepage}
  \lfoot{}
  \cfoot{}
  \rfoot{}
}
\renewcommand{\headrulewidth}{0.4pt}
%\renewcommand{\footrulewidth}{0.4pt}


\newrobustcmd*{\vocab}[1]{\emph{#1}}
\newrobustcmd*{\latin}[1]{\textit{#1}}

%%%%%%%%%%%%%%%%%%%%%%%%%%%%%%%%%%%%%%%%
% Customize list enviroonments
%%%%%%%%%%%%%%%%%%%%%%%%%%%%%%%%%%%%%%%%
% package to customize three basic list environments: enumerate, itemize and description.
\usepackage{enumitem}
\setitemize{noitemsep, topsep=0pt, leftmargin=*}
\setenumerate{noitemsep, topsep=0pt, leftmargin=*}
\setdescription{noitemsep, topsep=0pt, leftmargin=*}

%%%%%%%%%%%%%%%%%%%%%%%%%%%%%%%%%%%%%%%%
% Some really basic macros.
% (Lots of them were stolen from HoTT/Book.)
% See macros.tex in HoTT/book.
%
% This is a mess.  Needs clean-ups.
%%%%%%%%%%%%%%%%%%%%%%%%%%%%%%%%%%%%%%%%
\newrobustcmd*{\ctx}{\Gamma}
\newrobustcmd*{\entails}{\vdash}

\newrobustcmd*{\judgmentfont}[1]{{\normalfont\sffamily #1}}
\newrobustcmd*{\postfixjudgment}[1]{%
  \relax\ifnum\lastnodetype>0\mskip\medmuskip\fi
  \text{\judgmentfont{#1}}%
}
\newrobustcmd*{\prop}{\postfixjudgment{prop}}
\newrobustcmd*{\true}{\postfixjudgment{true}}
\newrobustcmd*{\type}{\postfixjudgment{type}}
\newrobustcmd*{\context}{\postfixjudgment{ctx}}

\newrobustcmd*{\truth}{\top}
\newrobustcmd*{\conj}{\wedge}
\newrobustcmd*{\disj}{\vee}
\newrobustcmd*{\falsehood}{\bot}
\newrobustcmd*{\imp}{\supset}
\newrobustcmd*{\zero}{0}

%%% Judgmental equality
\newrobustcmd*{\jdeq}{\equiv}
%%% Definition
\newrobustcmd*{\defeq}{\vcentcolon\equiv}
%%% Binary sums
\newrobustcmd*{\inlsym}{{\mathsf{inl}}}
\newrobustcmd*{\inrsym}{{\mathsf{inr}}}
\newrobustcmd*{\inl}{\ensuremath\inlsym\xspace}
\newrobustcmd*{\inr}{\ensuremath\inrsym\xspace}
%%% Booleans
\newrobustcmd*{\ttsym}{{\mathsf{tt}}}
\newrobustcmd*{\ffsym}{{\mathsf{ff}}}
%%% Pairs
\newrobustcmd*{\pair}{\ensuremath{\mathsf{pair}}\xspace}
\newrobustcmd*{\tuple}[2]{(#1,#2)}
\newrobustcmd*{\proj}[1]{\ensuremath{\mathsf{pr}_{#1}}\xspace}
%% Empty type
\newrobustcmd*{\abort}[1]{\ensuremath{\mathsf{abort}_{#1}}}
%%% Path concatenation
\newrobustcmd*{\concat}{%
  \mathchoice{\mathbin{\raisebox{0.5ex}{$\displaystyle\centerdot$}}}%
  {\mathbin{\raisebox{0.5ex}{$\centerdot$}}}%
  {\mathbin{\raisebox{0.25ex}{$\scriptstyle\,\centerdot\,$}}}%
  {\mathbin{\raisebox{0.1ex}{$\scriptscriptstyle\,\centerdot\,$}}}
}
%%% Transport (covariant)
\newrobustcmd*{\trans}[2]{\ensuremath{{#1}_{*}\mathopen{}\left({#2}\right)\mathclose{}}\xspace}
% Natural numbers objects
\newrobustcmd*{\Nat}{\mathsf{Nat}}
\newrobustcmd*{\rec}{\ensuremath{\mathsf{rec}}\xspace}
% Sequence
\newrobustcmd*{\Seq}{\ensuremath{\mathsf{Seq}}\xspace}
% Identity type
\newrobustcmd*{\Id}[1]{\ensuremath{\mathsf{Id}_{#1}}\xspace}
% Reflection
\newrobustcmd*{\refl}[1]{\ensuremath{\mathsf{refl}_{#1}}\xspace}
\newrobustcmd*{\J}{\ensuremath{\mathsf{J}}\xspace}

% fst,snd,case,id
\newrobustcmd*{\fst}{\textsf{fst}}
\newrobustcmd*{\snd}{\textsf{snd}}
\DeclareMathOperator{\case}{\textsf{case}}
\DeclareMathOperator{\caseif}{\textsf{if}}
\DeclareMathOperator{\casesplit}{\textsf{split}}
\DeclareMathOperator{\ttrue}{\textsf{tt}\xspace}
\DeclareMathOperator{\ffalse}{\textsf{ff}\xspace}
\newrobustcmd*{\id}{\textsf{id}}

\newrobustcmd*{\op}[1]{\operatorname{#1}}

\newrobustcmd*{\universe}{\mathcal{U}}

%inductive types
\newrobustcmd*{\ind}{\ensuremath{\mathsf{ind}}\xspace}
%higher inductive types
%interval
\newrobustcmd*{\interval}{\ensuremath{I}\xspace}
\newrobustcmd*{\seg}{\ensuremath{\mathsf{seg}}\xspace}
%circle
\newrobustcmd*{\Sn}{\mathbb{S}}
\newrobustcmd*{\base}{\ensuremath{\mathsf{base}}\xspace}
\newrobustcmd*{\lloop}{\ensuremath{\mathsf{loop}}\xspace}


\usepackage{proof-dashed}
\usepackage{tikz-cd}

\metadata{IPL Notes}{Feb 28, 2014}
\begin{document}
This note summarizes the rules of inference for \ac{IPL} as I understand them.
This is based on the lectures given by Professor Robert Harper in September
2013 at CMU~\cite{Harper2012}. 
Notes for Harper's lectures were transcribed by his students and
this summary is based on the recorded lectures and the notes. 

As advanced by Per Martin-L\"{o}f, a modern presentation of \ac{IPL}
distinguishes the notions of \vocab{judgment} and \vocab{proposition}. A
judgment is something that may be known, whereas a proposition is something that
sensibly may be the subject of a judgment. For instance, the statement ``Every
natural number larger than $1$ is either prime or can be uniquely factored into
a product of primes\@.'' is a proposition because it sensibly may be subject to
judgment. That the statement is in fact true is a judgment.
Only with a proof, however, is it evident that the judgment indeed holds.

Thus, in \ac{IPL}, the two most basic judgments are $A \prop$ and $A \true$:
\begin{alignat*}{2}
  A \prop &&\quad& \text{$A$ is a well-formed proposition} \\
  A \true &&& \text{\begin{tabular}[t]{@{}l@{}}
                Proposition $A$ is intuitionistically true, i.e., has a proof.
              \end{tabular}}
\end{alignat*}
The inference rules for the $\prop$ judgment are called \vocab{formation rules}.
The inference rules for the $\true$ judgment are divided into classes:
\vocab{introduction rules} and \vocab{elimination rules}. 

The meaning of a proposition $A$ is given by the introduction rules for the
judgment $A \true$. The elimination rules are dual and  describe what may be
deduced from a proof of $A \true$.  The principle of \vocab{internal coherence},
also known as \emph{Gentzen's principle of inversion}, is that the introduction
and elimination rules for a proposition $A$ fit together properly.  The
elimination rules should be strong enough to deduce all information that was
used to introduce $A$ (\vocab{local completeness}), but not so strong as to
deduce information that might not have been used to introduce $A$ (\vocab{local
  soundness}). 

\medskip 

\hrule

\begin{center}
CONJUNCTION
\end{center}
\begin{itemize}
\item[(formation)] 
If $A$ and $B$ are well-formed propositions, then so is
their \emph{conjunction}, which we write as $A \conj B$.
\begin{equation*}
  \infer[{\conj}\mathsf{F}]{A \conj B \prop}{
    A \prop & B \prop}
\end{equation*}

\item[(introduction)]
%\paragraph{Introduction.}
To give meaning to conjunction, we must say how
to introduce the judgment $A \conj B \true$.
A verification of $A \conj B$ requires a proof of $A$ and
a proof of $B$.
\begin{equation*}
  \infer[{\conj}\mathsf{I}]{A \conj B \true}{
    A \true & B \true}
\end{equation*}

\item[(elimination)]
%\paragraph{Elimination.}
Because every proof of $A \conj B$ comes from a pair of proofs, one of $A$ and
one of $B$, we are justified in deducing $A \true$ and $B \true$ from a proof of
$A \conj B$: 
\begin{mathpar}
  \infer[{\conj}\mathsf{E}_1]{A \true}{
    A \conj B \true}
  \and
  \infer[{\conj}\mathsf{E}_2]{B \true}{
    A \conj B \true}
\end{mathpar}
\end{itemize}

\medskip 

\hrule
%\newpage
\begin{center}
TRUTH
 \end{center}
%We denote by $\truth$ the \emph{trivially true} proposition. 
\begin{itemize}
\item[(formation)]
The formation rule serves as immediate evidence for the judgment that $\truth$ is a
well-formed proposition.
\begin{equation*}
  \infer[{\truth}\mathsf{F}]{\truth \prop}{
    }
\end{equation*}

\item[(introduction)]
Since $\truth$ is a trivially true proposition, its introduction rule makes the
judgment $\truth \true$ immediately evident.

\begin{equation*}
  \infer[{\truth}\mathsf{I}]{\truth \true}{
    }
\end{equation*}

\item[(elimination)]
Since $\truth$ is trivially true, an elimination rule should not increase
our knowledge---we put in no information when we introduced $\truth \true$, so,
by the principle of conservation of proof, we should get no information out. Thus,
there is no elimination rule for $\truth$.
\end{itemize}

%% \medskip 

%% \hrule
\newpage

\begin{center}
ENTAILMENT
 \end{center}
\emph{Entailment} is a judgment and is written as 
\begin{equation*}
  A_1 \true, \dotsc, A_n \true \entails A \true
\end{equation*}
This expresses the judgment that $A \true$ follows from 
$A_1 \true, \dotsc, A_n \true$. 
One can view $A_1 \true, \dotsc, A_n \true$ as being assumptions from which
the conclusion $A \true$ may be deduced. 
We assume that the entailment judgment satisfies several \emph{structural
  properties}: reflexivity, transitivity, weakening, contraction, and
permutation. 
\begin{itemize}
\item[Reflexivity:] An assumption is enough to conclude the same judgment.
\begin{equation*}
  \infer[\text{\textsf{R}}]{A \true \entails A \true}{
    }
\end{equation*}

\item[Transitivity:]
If you prove $A \true$, then you are justified in using it in a proof.
\begin{equation*}
  \infer[\text{\textsf{T}}]{C \true}{
    A \true &
    A \true \entails C \true}
\end{equation*}

Reflexivity and transitivity are undeniable since without them it would be
unclear what is meant by an \emph{assumption}.  An assumption should be strong enough
to prove conclusions (reflexivity), and only as strong as the proofs they stand for
(transitivity). 
The remaining structural properties---weakening, contraction, and
permutation---could be denied.  Logics that deny any of these properties are
called \emph{substructural logics}. 

\item[Weakening:]
We can add assumptions to a proof without invalidating that proof.
\begin{equation*}
  \infer[\text{\textsf{W}}]{B \true \entails A \true}{
    A \true}
\end{equation*}
\item[Contraction:]
The number of copies of an assumption does not matter.
\begin{equation*}
  \infer[\text{\textsf{C}}]{A \true \entails C \true}{
    A \true, A \true \entails C \true}
\end{equation*}
\item[Permutation:]
aka ``exchange;'' the order of assumptions does not matter.
\begin{equation*}
  \infer[\text{\textsf{P}}]{\pi(\ctx) \entails C \true}{
    \ctx \entails C \true}
\end{equation*}
\end{itemize}


\medskip 

\hrule

\begin{center}
IMPLICATION
\end{center}
\begin{itemize}
\item[(formation)] 
\begin{equation*}
  \infer[{\imp}\mathsf{F}]{A \imp B \prop}{
    A \prop & B \prop}
\end{equation*}
\item[(introduction)]
\begin{equation*}
  \infer[{\imp}\mathsf{I}]{A \imp B \true}{
    A \true \entails B \true}
\end{equation*}
In this way, implication internalizes the entailment judgment as a proposition,
while we nonetheless maintain the distinction between propositions and
judgments.
%% (As an aside for those familiar with category theory, the relationship between
%% entailment and implication is analogous to the relationship between a mapping
%% and a collection of mappings internalized as an object in some category.) 
\item[(elimination)]
\begin{equation*}
  \infer[{\imp}\mathsf{E}]{B \true}{
    A \imp B \true & A \true} \,.
\end{equation*}
This rule is sometimes referred to as \latin{modus ponens}.
\end{itemize}

%% \medskip

%% \hrule
\newpage
\begin{center}
DISJUNCTION
\end{center}
\begin{itemize}
\item[(formation)]
\begin{equation*}
  \infer[{\disj}\mathsf{F}]{A \disj B \prop}{
    A \prop & B \prop}
\end{equation*}
\item[(introduction)]
\begin{mathpar}
  \infer[{\disj}\mathsf{I_1}]{A \disj B \true}{
    A \true}
  \and
  \infer[{\disj}\mathsf{I_2}]{A \disj B \true}{
    B \true}
\end{mathpar}
\item[(elimination)]
\begin{equation*}
  \infer[{\disj}\mathsf{E}]{C \true}{
    A \disj B \true &
    A \true \entails C \true & B \true \entails C \true}
\end{equation*}
\end{itemize}

\medskip 

\hrule
\begin{center}
FALSEHOOD
\end{center}
\begin{itemize}
\item[(formation)]
The unit of disjunction is falsehood, the proposition that is trivially never
true, which we write as $\falsehood$.  Its formation rule is immediate evidence
that $\falsehood$ is a well-formed proposition. 
\begin{equation*}
  \infer[{\falsehood}\mathsf{F}]{\falsehood \prop}{
    }
\end{equation*}
\item[(introduction)]
Because $\falsehood$ should never be true, it has no introduction rule.
\item[(elimination)]
\begin{equation*}
  \infer[{\falsehood}\mathsf{E}]{C \true}{
    \falsehood \true}
\end{equation*}
The elimination rule captures \latin{ex falso quodlibet}: from a proof of $\falsehood \true$, we may deduce that \emph{any} proposition $C$ is true because there is ultimately no way to introduce $\falsehood \true$.
Once again, the rules cohere.
The elimination rule is very strong, but remains justified due to the absence of any introduction rule for falsehood.
\end{itemize}

\medskip 

\hrule
%% \begin{center}
%% NEGATION
%% \end{center}
%% \begin{itemize}
%% \item[(formation)]
%% \item[(introduction)]
%% \item[(elimination)]
%% \end{itemize}

%% \nocite{Pfenning2009a, Pfenning2009b}
\begin{thebibliography}{1}


\bibitem{Harper2012}
Robert Harper.
\newblock Carnegie Mellon University course: 15-819 Homotopy Type Theory.
\newblock
  \url{http://www.cs.cmu.edu/~rwh/courses/hott/},
  Fall 2012.

\bibitem{Pfenning2009b}
Frank Pfenning.
\newblock Lecture notes on harmony.
\newblock
  \url{http://www.cs.cmu.edu/~fp/courses/15317-f09/lectures/03-harmony.pdf},
  September 2009.

\bibitem{Pfenning2009a}
Frank Pfenning.
\newblock Lecture notes on natural deduction.
\newblock
  \url{http://www.cs.cmu.edu/~fp/courses/15317-f09/lectures/02-natded.pdf},
  August 2009.

\end{thebibliography}

%% \bibliographystyle{plain}
%% \bibliography{hott_references}

\end{document}
